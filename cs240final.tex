\documentclass{article}

\usepackage{amsmath}
\usepackage{amsthm}
\usepackage{amsfonts}
\usepackage{amssymb}
\usepackage{verbatim}
\usepackage{algorithmicx}
\usepackage{algpseudocode}


\theoremstyle{plain}
\newtheorem{thm}{Theorem}
\newtheorem{cor}{Corollory}
\newtheorem{prop}{Proposition}


\theoremstyle{definition}
\newtheorem{eg}{Example}
\newtheorem{rmk}{Remark}
\newtheorem{defn}{Definition}

\setlength{\parindent}{0cm}

\begin{document}
\section{Dictionary}

\begin{defn}
    A dictionary/associative array is a abstract data type with (key, value) pair with
    operations
    \begin{itemize}
        \item Addition a pair
        \item Removal of a pair
        \item Modification of a existing pair
        \item Lookup a value of a existing pair
    \end{itemize}
\end{defn}

There are different implementation of this abstract data type such as
\begin{itemize}
    \item Tree including B-tree, AVL-tree, Red-Black-Tree
    \item Hash Function
    \item Direct Adressing
\end{itemize}

\subsection{Direct Addressing}
If $\forall k: 0\leq k\leq M$, where $M\in\mathbb{N}$ then we can use an
array $A[k]\leftarrow v$ to store the pair. Each operation is $\Theta(1)$, and
the total storage is $\Theta(M)$.

$Key = \{a_n\}^M_{n=0} \xrightarrow{A} Value)$

\subsection{Hashing}
For $k\in Any$, then define $h: Any\rightarrow \{0, ..., M\}$ to be a hash function.
We can store the key in an Array.

$Key = Any\xrightarrow{h}\{a_n\}^M_{n=0}\xrightarrow{A} (Key, Value)$

The hash function is generally not injective. We get a collision when we want to
insert a (key, value) pair into a bucket that is occupied.

\begin{defn}
    Load Factor: $\alpha = \frac{n}{M}$, where $n$ is the number pairs already stored,
    and $M$ is the array size
\end{defn}

\subsubsection{Chaining}
$Key = Any\xrightarrow{h}\{a_n\}^M_{n=0}\xrightarrow{A} List(Key, Value))$

\subsubsection{Linear Probing}
Hash Construction: $H \rightarrow H_L$\\
Define: $h_D(k, i) = h_1(k) + i \bmod{M}$

$Key = Any\times\mathbb{N}\xrightarrow{h_L}\{a_n\}^M_{n=0}\xrightarrow{A} (Key, Value)$

It will through iterate $0, 1, ...$ until $(Key, Value)$ is found.

\begin{eg}
    $h_L(k, i) = (h(k) + i)$
\end{eg}

\subsubsection{Double Hashing}
Hash Construction: $H_1\times H_2\rightarrow H_D$\\
Define: $h_D(k, i) = h_1(k) + ih_2(x) \bmod{M}$

$Key = Any\times\mathbb{N}\xrightarrow{h_D}\{a_n\}^M_{n=0}\xrightarrow{A} (Key, Value)$

\subsubsection{Cuckoo Hashing}
We can look at Cuckoo Hashing as a bipartite graph, $G = (U, V, E)$, where $U$ and $V$
are array of size $M$. The occupied buckets in $U$ are $h_1(k)$, and $V$, $h_2(k)$
for some $k$. The edges are ($h_1(k), h_2(k)$) for some $k$. Remember each slot can only
have one element.

\begin{algorithmic}[1]
    \Function{CuckooHashing}{A, key, value}
    \State pos = $h_1(k)$:
    \While{A[pos] is not None}
        \State A[pos], key = key, A[pos]
        \If{$h_1(k)$ != pos}
            \State pos = $h_1(k)$
        \Else
            \State pos = $h_2(k)$
        \EndIf
    \EndWhile
    \State A[pos] = (key, value)
    \EndFunction
\end{algorithmic}

\begin{thm}
    The insertion of a key succeeds if the connected component containing
    $key$ contains either no cycles or only one cycle. (pseudotree)
\end{thm}

\section{Range Searching}
\subsection{QuadTree}
\begin{algorithmic}
    \State QuadTree:: Leaf Point $\mid$ Branch QuadTree QuadTree QuadTree QuadTree
\end{algorithmic}

\subsection{BST Range Search}
Given $[a, b]$, and a BST, the problem is to find $\{v\in Nodes(T)\mid a <= v <= b\}$,
then for a subtree, three cases may happen by the law of triochotomy:
\begin{itemize}
    \item If $b < root(T)$,
        \begin{gather*}
            \forall v\in Nodes(T_{right}): v \geq root(T)\\
            \nexists v\in Nodes(T_{right}): v < root(T)\\
            \nexists v\in Nodes(T_{right}): v < b
        \end{gather*}
        Therefore, no search to the left.
    \item If $root(T) < a$
        \begin{gather*}
            \forall v\in Nodes(T_{left}): v \leq root(T)\\
            \nexists v\in Nodes(T_{right}): v > root(T)\\
            \nexists v\in Nodes(T_{right}): v > a
        \end{gather*}
        No search to left.
    \item If $a \leq root(T) \leq b$, search left and right.
\end{itemize}

There are tree type of nodes
\begin{itemize}
    \item Boundary nodes: No constraints, or contrainst not dominating $a \leq v \leq b$
    \item Inside nodes: There exists constraints dominating $[a,b]$,
        the topmost is also called allocation node
    \item Outside nodes: Nodes that are being pruned
\end{itemize}

\subsection{Range Tree}
To perform a range query where $A = [a_{11}, a_{12}] \times ... \times [a_{n1}, a_{n2}]$,
\begin{itemize}
    \item Boundary node: since it's not dominating, we check node if in range
    \item Allocation node: Dominating, construct a BST with all the points of the subtree
        based on the next coordinates. Then do BST.
\end{itemize}
$T$ construct a binary based on $Node(T)$

\subsection{Conclusion}
Range Tree:
\begin{itemize}
    \item Fastest: $O(s + n(\log n)^d)$
    \item Space: $O(n(\log n)^{d-1})$
    \item Complicated insert/deletion
\end{itemize}

kd-Tree:
\begin{itemize}
    \item $O(s + n)$
    \item Space: $O(n)$
    \item Duplicate case to be taken care of
\end{itemize}

\section{String Matching}
\subsection{Knuth-Morris-Pratt algorithm}
Using DFA, go to the state where it has failed. There is the failure array and the DFA.

State $k$ represents that the last $k$ elements are matching ($P[:k]$),
we want to parse the next element $k+1$-th element ($P[k]$). If success, move on next
state. If failure, jump to the state where most the suffix matches the current good suffix,
and still try to parse the next character. (no parsing this stage)

The main mapping is a prefix of $P$, of prefixes.

\subsection{Boyer-Moore Algorithm}
There are two main components to this algorithms:
\begin{itemize}
    \item Bad character rule: Shift until matches the character
    \item Good suffix rule: Shift where the suffix matc
\end{itemize}
You have $P, T, k$, then for $P[k:]$ suffix, compare to $T$, we want to get the maximum
number of matching suffix of this suffix, and bad character index. $P[k:][l:]$, and
$P[l]$ where $l$ indicate the bad character index. Take the difference of $n-l$, it's
the amount you have to shift. (proof later)


$k = k+1$

Good suffix same idea as KMP, then find the main mapping is suffix of suffix. The reverse
of KMP, if failed then to KMP on the suffix with long with the suffix (weak suffix rule)

\section{Encoding}
Decoding: Greedy\\
Encoding: Merge frequencies

\subsection{RLE Encoding}
Decoding: First Bit, determines $0,1,0,1,...$ or $1, 0, 1, 0,...$ alternation\\
Number of zero is the image of $\log k$, so a mapping $\log k + 1$ gives us the length.

\subsection{LZW Encoding}
Encoding: Encode in a greedy way, upon failure, encoding, and add in dictionary\\
Decoding: Decode while building, upon seeeing unknown character, use the
the $previous + previous[0]$, and assign number to the dictionary

\subsection{BWT Encoding}
Idea: adjacent character maps to each other, then stable sort. to recover the values.



\section{Appendix}
\subsection{Median}
    Median of a list of length $n$ and index starting at 1 is defined in statistics:
    \begin{itemize}
        \item If $n$ is even, Median = $\frac{a_{\frac{n}{2}} + a_{\frac{n+1}{2}}}{2}$
        \item If $n$ is odd, Median = $a_{\frac{n+1}{2}}$
    \end{itemize}
    Note that $a_k$ means that
    \begin{itemize}
        \item There are $k-1$ elements before $k$-th index excluding itself. We can see $k$
            as fence, and we can assign every left pole of a fence with the fence value.
            We want to know the number of fences between two poles, so we do substract.
        \item There are $(n+1) - (k+1) = n-k$ after $k$-th index excuding itself.
            Same logic, we take the right pole of fence $n$ and $k$, and do substraction.
    \end{itemize}
    The intuitive sense of the median is to have same amount of elements before and after
    $k$-th excluding $k$ itself. Thus, we want to find $k$ such that
    \begin{gather*}
        k-1 = n-k\\
        k + k = n+1\\
        2k = n+1
    \end{gather*}
    If $n$ is odd, then we want the element at index,
    \begin{gather*}
        k = \frac{n+1}{2}
    \end{gather*}
    If $n$ is even, then there is no good way of splitting, then we want to find two
    adjacent elements where the number of elements before first elements is same
    as the elements after second elements. Let $k$ be the index of first element.
    \begin{gather*}
        k-1 = (n+1)-((k+1)+1)\\
        k-1 = n-k-1\\
        k = \frac{n}{2}
    \end{gather*}
    If $k = \frac{n}{2}$, then $k-1 = n-k-1$ holds, therefore we have $k$-th and $k+1$-th
    element, and we take the average of the to give a single number.

\subsection{Binary Tree}
One way to define a binary tree is as follows

We can define operation as $root(T), subtrees(T), nodes(T)$

\subsubsection{Binary Search Tree}
    \begin{algorithmic}
        \State BinarySearchTree:: Empty $\mid$ Value BinarySearchTree BinarySearchTree
    \end{algorithmic}
    A Binary search Tree is a Binary Tree with this addition property:
    \begin{itemize}
        \item $Empty$ is a Binary Search Tree
        \item $T$ is a Binary Search Tree if
            $T_{left}$ and $T_{right}$ are Binary Search Trees, and
            \begin{gather*}
                \forall v\in Nodes(T_{left}): v \leq root(T)\\
                \forall v\in Nodes(T_{right}): v \geq root(T)
            \end{gather*}
    \end{itemize}

\subsection{Interval}
    Recall the law of trichotomy: Let $<$ be an order relation in $X$, then
    \begin{align*}
        \forall x, y\in X ((x < y) \lor (y > x) \lor (x = y))
    \end{align*}

    Given $[a, b]$ and $v$ where $a \leq b$, then
    \begin{gather*}
        (v < a) \lor (a\leq v\leq b) \lor (b > v)
    \end{gather*}

\subsection{Graph Theory}
A graph can be described, and constructed based on
\begin{itemize}
    \item The edges
    \item The nodes, and their adjacent nodes
\end{itemize}
As geometric shape can also be seen as such.

\subsection{List}
Here are some useful constructions of lists/strings of length $n$. The $k$-th index
can be seen as a fence or a pole
\begin{itemize}
    \item $\{P[k]: k\in\{a_n\}^{n-1}_{i=0}\}$: list of characters
    \item $\{P[k:]: k\in\{a_n\}^{n-1}_{i=0}\}$: list of suffixes
    \item $\{P[:k]: k\in\{a_n\}^{n}_{i=1}\}$: list of prefixes
    \item Map
    \item Foldr
\end{itemize}
Notice slicing is defined as left pole. Here are different interpretation of the function
\begin{itemize}
    \item $P[:k]$: there are $k$ element in the prefix, the prefix before $k+1$-th element
    \item $P[k]$: there are $k$ element before this $k+1$-th element
    \item $P[k:]$: there are $k$ element before this suffix
\end{itemize}

\subsubsection{Shifting}
Problem: We have fence $i$, fence $j$, we want to shift fence $i$ until $i$ reaches
fence $j$. How many fence do we have to shift?\\
$i + n$ adds $n$ poles between $i$-th pole and $i+n$-th pole. In other words,
\begin{gather*}
    \exists n: i + n = j\\
    n = j - i
\end{gather*}
Affine map/Group action
\end{document}
