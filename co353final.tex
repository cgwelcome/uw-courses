\documentclass{article}

\usepackage{amsmath}
\usepackage{amsthm}
\usepackage{amsfonts}
\usepackage{amssymb}
\usepackage{verbatim}
\usepackage{algorithmicx}
\usepackage{algpseudocode}


\theoremstyle{plain}
\newtheorem{thm}{Theorem}
\newtheorem{cor}{Corollory}
\newtheorem{prop}{Proposition}


\theoremstyle{definition}
\newtheorem{eg}{Example}
\newtheorem{rmk}{Remark}
\newtheorem{defn}{Definition}

\setlength{\parindent}{0cm}

\begin{document}
\begin{thm}
    A problem $A$ is polynomial-time reducible to problem $B$, denoted $A\leq_P B$,
    if all instance of $A$ can be reduced to $B$.
\end{thm}

Parallel to set theory: $A\subseteq B$, $\forall a\in A, a\in B$

Let $A\leq_P B$, if I can solve $B$ in polynomial time, then I can solve $A$ in polynomial
time.


\begin{thm}
    A problem $P$ is a complexity class in $NP$ such that there is a polynomial time to
    solve them
\end{thm}

\begin{gather*}
    P\subseteq NP\\
    \forall Q\in P, Q\in NP
\end{gather*}


\begin{thm}
    A problem $H$ is NP-complete if for all $q\in$ NP are polynomially reducible to
    $H$.
\end{thm}

In notation,
\begin{gather*}
    H\in NP \land \forall Q\in NP: Q\leq_P H\\
\end{gather*}
\end{document}
