\documentclass{article}

\usepackage{amsmath}
\usepackage{amsthm}
\usepackage{amsfonts}
\usepackage{verbatim}

\newtheorem{thm}{Theorem}
\newtheorem{defn}{Definition}
\newtheorem{eg}{Example}
\newtheorem{cor}{Corollory}

\newcommand{\floor}[1]{\lfloor #1 \rfloor}
\newcommand{\ceiling}[1]{\lceil #1 \rceil}

\setlength{\parindent}{0em}

\begin{document}
Decoding strategies:

IMLD decodes $r\in A^n$ to the neareest neighbour; i.e. it finds
$c\in C$ which maximizes $d(c, r)$. equivalently, IMLD maximizes
$P(r \text{is received} | c \text{is sent})$. It makes more sense
to use Minimum Error Decoding which decodes to $c\in C$ that maximizes
$P(c \text{is sent} | r \text{is received})$

\begin{eg}
   Suppose I'm sending letter over a channel to convey English words. Then
    not all symbols are equally likely to be sent
\end{eg}

Consider $c = \{c_1 = 000, c_2 = 111\}$. Let $P(c_i)$ denote the probability
that $c_i$ is being sent, and $P(c_1) = 0.1$ and $P(c_2) = 0.9$.
Using a BSC, with $p = \frac{1}{4}$, suppose $r=100$ is received.

\begin{enumerate}
    \item Using IMLD, we decode to nearest neighbour $c_1 = 000$
    \item Using MED, we
        \begin{align*}
            P(c_1| r) &= P(r|c_1)P(c_1)/P(r)\\
            &= 0.25*0.75*0.75 0.1/P(r) = 9/640 * 1/P(r)\\
            P(c_2| r) &= P(r|c_2)P(c_2)/P(r)\\
            &= 0.75*0.25*0.25 0.9/P(r) = 27/640 * 1/P(r)\\
        \end{align*}
\end{enumerate}

Note: In practice, MLD is used. We will mostly focus on IMLD or CMLD.
Also, in practice MLD and MED often coincide.

Next topic: Linear code.
We will use linear algebra to design codes. Instead of using scalar
$\mathbb{R}$, we will use scalars from a finite field.

\subsection{Finite Field}
A field $(F, +, *)$ consists of a set $F$, an addition operation,
$+: F\times F\rightarrow F$ and a multiplication operation: $F\times F\rightarrow F$.
Not a trivial ring.

\begin{defn}
    A finite field is a field $\mathbb{F}$ which has many elements.
\end{defn}

\begin{eg}
    \mathbb{Z}_p, we only need to show that $p\not\mid a$, then $a$ has a multiplicative
    inverse modulo $p$.
\end{eg}

Recall: Bezout's lemma says that given $a,b\in\mathbb{Z}, \exists x,y\in\mathbb{Z}$
such that $ax + by = gcd(a,b)$.

By Bezout lemma, $\exists x,y$ such that $ax + py = 1$ since $gcd(a, p) = 1$. So
reducing modulo $p$, we get $[a][x] = [1]$, so $[x] = [a]^{-1}$.

Note: This can be made effective using the extended Euclidean Algorithm.

TODO: Review EEA, etc. from MATH 135.

Exercise: If $n$ is not prime, show that $\mathbb{Z}_n$ is not a field. If $m$ is a
proper divisor of $n, 1 < m < n$, then $[m]$ has no inverse in $\mathbb{Z}_n$

Why is $\mathbb{Z}_5[i]$ not a field?

We note that -1 already has a square root: $2^2 = -1$ in $\mathbb{Z}_5$, and thus
$(2+i)(2-i) = 0$ but these elements $2+i, 2-i$ are non-zero, they be invertible.
This gives a contradiction that $2+i = 0$.

\begin{defn}
    The order of a field $\mathbb{F}$ is the number of elemnts of that fields.
\end{defn}

e.g. $\mathbb{Z}_5$ is a field of order 5, $\mathbb{Z}_7[i]$ is a field of order
49.

Question: We can see that there is a finite field of order $g$ for any prime $p$, namely
$\mathbb{Z}_p$. What about fields with $n$ elements where $n$ is composite?

\subsection{Characteristic of a finite field}
Suppose $\mathbb{F}$ is any finite field, and consider the sequence,
$1, 1+1, 1+1+1,...$  where 1 is the multiplicative identity. We will notate these
as if they where integers.

We remark that since these all elements of the finite set $\mathbb{F}$, there must
be eventually be a repetition $m=n$ for some integer $0\leq m < n$. Then note that
we can substract $m$ from each side to get $0 = n - m$, $0 = 1+1+1+...$

So now we've established that there is some positive integer $K$  such that $K = 0$ in
$\mathbb{F}$.

\begin{defn}
    Let $\mathbb{F}$ be any field. The characteristic of $\mathbb{F}$  is the smallest
    integer with $k\in\mathbb{Z}_{> 0}$ such that $k=0$ in $\mathbb{F}$ in $\mathbb{F}$.
    This $k$ is denoted $char(F)$.
\end{defn}

\begin{eg}
   $\mathbb{Z}_7[i]$ has characteristic 7.If there is no characteristic, we say
    $char(\mathbb{F}) = 0$
\end{eg}

\begin{thm}
    The characteristic of any finite field is a prime number.
\end{thm}

\begin{proof}
    Suppose that the characteristic was $m=ab$ for $a, m$.
    The $ab = 0$ in $\mathbb{F}$. Note that $a\ne 0$, since $a<m$.,
    $b = 0, b < m$.  Contradiction.
\end{proof}

$\mathbb{Z}_p$ inside of $\mathbb{F}$. Suppose $\mathbb{F}$ is a field of
characteristic $p>0$. The nconsider the set $\mathbb{Z} = \{0,1,2,3,4, ..., p-1\} \subset{\mathbb{F}}$,
A field inside of another field. It is isomorphic to $\mathbb{Z}_p$.

\begin{thm}
    If $\mathbb{F}$ has char. $p$, then $\mathbb{Z}_p$ is a subfield of $\mathbb{F}$.
\end{thm}

Fact: $\mathbb{F}$ is in fact a vector space over $\mathbb{Z}_p$.

Finite field as vector spaces over $\mathbb{Z}_p$. Now that we know that
$\mathbb{Z}_p$-vector space, let's use linear algebra to more about
the structure of $\mathbb{F}$

We start with a basis. $B = {b_1, ..., b_k}$ for $\mathbb{F}$ where
$k = dim \mathbb{F}$ over $\mathbb{Z}_p$. By linear algebra, every element of
$\mathbb{F}$ can be uniquely expressed as... (definition)

\begin{thm}
    If $\mathbb{F}$ is a finite field of char $p$, then $\mathbb{F} = p^k$, for
    some $k\in\mathbb{Z}_{> 0}$. In other words, any finite field has an order of that
    prime power.
\end{thm}


Does a field of order $p^n$ exists for any $p$ and $n$?

Answer: The answer is yes! We will construct them using the theory of polynomial rings.

\begin{eg}
    Over $\mathbb{Z}_p$ we can construct a field of order 49 by adjoining a root of the
    irreducible polynomial $x^+1 \in \mathbb{Z}_p[x]$. This gives $\mathbb{Z}_7[i]$, which
    has order 49. Note: $x^2+1\in\mathbb{Z}_5[x]$. is not irreducible.
\end{eg}

\subsection{Polynomial Rings}
Let $\mathbb{F}$ be a field. Then
$\mathbb{F}[x] = \{a_0 + a_1x + a_2x^2 + ... + a_nx^n: a_0, ... a_n\in\mathbb{F},
n\in\mathbb{Z}_{\geq 0}\}$. The polynomial over $\mathbb{F}$.
We can add/subtract/multiply these as normal.

\begin{eg}
    $\mathbb{Z}_7[x]$ is the set of polynomial with coefficients in $\mathbb{Z}_7$.
    \begin{align*}
        (2+3x+4x^2)(3+x) &= 6 + 2x + 9x + 3x^2 + 12x^2 + 4x^3\\
        &= 6 + 4x + x^2 + 4x^3
    \end{align*}
\end{eg}

\begin{thm}[Division Algorithm]
    For any $a(x), b(x)\in\mathbb{F}[x]$ with $b(x)\ne 0$, there exist unique
    polynomial $q(x), r(x)$ such that
    $a(x) = q(x)b(x) + r(x)$ and $deg r < deg b$, $r = 0$
\end{thm}

\begin{eg}
    In $\mathbb{Z}_2[x]$, we can effectively find $q$ and $r$ using long division
    If $a(x) = 1+x^2+x^3+x^5, b(x) = 1 + x^3$,
\end{eg}

Idea for constructing finite field
\begin{enumerate}
    \item For $\mathbb{Z}_p$: start by looking at $\mathbb{Z}$, then find
        prime(or irreducible) $p\in\mathbb{Z}$ and construct
        $\mathbb{Z}\setdiff p\mathbb{Z}$, which is the set of equivalence classes
        of integers under relation $a\equiv b\bmod{p}$
    \item Next, we look at $\mathbb{F}[x]$, and do a similar construction. So we find
        an irreducible $f(x)\in\mathbb{F}[x]$ then consider the relation
        $a(x)\equiv b(x)\bmod{f(x)}$, and let $\mathbb{F}[x]\bmod{f(x)}$ be
        the set of equivalence classes. This will be a field!
\end{enumerate}

\begin{defn}
    For $a(x), b(x)\in\mathbb{F}[x]$ with $b\ne 0$, we say $b(x)$ divides $a(x)$
    or $b(x)\mid a(x)$ if the remainder of $a(x)\div b(x)$ is 0
\end{defn}

\begin{defn}
    For $f(x)\in\mathbb{F}[x]\setdiff\{0\}$ and $g(x), h(x)\in\mathbb{F}[x]$, we say
    "g(x) is congruent to $h(x)$ modulo $f(x)$" or $g(x)\mid g(x) - h(x)$.
\end{defn}

Fact: For a fixed $f(x)$, this relation is an equivalence relation. That means that is
\begin{enumerate}
    \item Reflexive: $g(x)\equiv g(x) \bmod{f(x)}$
    \item Symmetric: $g(x)\equiv h(x) \bmod{f(x)}$ iff $h(x)\equiv g(x)\bmod{f(x)}$
    \item Transitive
\end{enumerate}

\begin{thm}
    Suppose $f(x)\in\mathbb{F}[x]$ has degree $n\geq 0$. Then
    $R = \{a_0+ a_1x + ... + a_{n-1}x^{n-1}\}$ has properties that
    \begin{enumerate}
        \item Every element of $\mathbb{F}[x]$ is congruent to an element of $R$
            modulo $f(x)$.
        \item For every choice of $a_0,..., a_{n-1}$ get a distinct polynomial
            $a_0 + a_1x + ... + a_{n-1}x^{n-1}$
    \end{enumerate}
\end{thm}

Proof sketch: Follows from the division algorithm.

\subsection{Polynomial Rings "quotiented" by $<f(x)>$}}
Given $f(x)\in\mathbb{F}[x]$ with $deg f = n \geq 0$, we define $\mathbb{F}[x]\setdiff
<f(x)>$ to be the set of equivalence classes of polynomial modulo $f(x)$.
$F[x]\setdiff <f(x)> = \{[a_0 + ... + a_{n-1}x^{n-1}: a_0, ..., a_{n-1}\in\mathbb{F}\}$.

Question: If $|\mathbb{F}| = q$, then what is $|\mathbb{F}[x]\setdiff <f(x)>$|?
Answer: $q^n$,

Lemma:$\mathbb{F}[x]\setdiff <f(x)>$ is a commutative ring.

\begin{thm}[Freshman's dream]
    In $\mathbb{Z}_p[x]$, $(a+b)^p \equiv a^p + b^p \bmod{p}$
\end{thm}

Is that ring a field?

Yes! This is because $x^3+x+1$ is irreducible, as we will see.


Find the inverse of $x^2 + 1\in\mathbb{Z}_2[x]\setddif <x^3+x+1>$
Solution: We do a similar process to finding inverse in $\mathbb{Z}_n$ namely, we
run the EEA.

\begin{defn}
    For $f(x)\in\mathbb{F}[x]$ of degree $\geq 1$, we say $f$ is irreducible if
    whenever $f(x) = g(x)h(x)$ in $\mathbb{F}[x]$. We have either
    $deg g = 0$ or $deg h = 0$
\end{defn}

\begin{eg}
    Any $deg 1$ polynomial such that $2x+1\in\mathbb{Z}_7[x]$ is irreducible.
\end{eg}

\begin{eg}
    Any degree 2 or 3 polynomial is irreducible iff it has no roots. So
    $x^3 + x + 1\in\mathbb{Z}_2[x]$ is irreducible since $0^3 + 0 + 0 \ne 0$ and
    $1^3 + 1 + 1 \ne 0$
\end{eg}

\begin{defn}
    A polynomial $f(x)$ is monic if the leading coefficient is 1.
\end{defn}

\begin{eg}
    $x^3 + x + 1\in\mathbb{Z}_2[x]$ is monic, $2x+1\in\mathbb{Z}_7[x]$ is not
\end{eg}

\begin{thm}
    Any nonzero $f(x)\in\mathbb{F}[x]$ ca nbe uniquely factored as
    $f(x) = cf_1(x)f_2(x)...f_k(x)$ where $c\in\mathbb{F}$, $f_i(x)$ is
    monic irreducible, up to reordering of the terms.
\end{thm}

\begin{thm}
    $\mathbb{F}[x]\setdiff<f(x)>$ is a field iff $f$ is irreducible.
\end{thm}

\begin{proof}
    For any $f(x)\in\mathbb{F}[x]$, this construction gives a commutivative ring, so
    we only need to check for existance of multiplicative inverse. So suppose
    $g(x)\in\mathbb{F}[x]\setdiff<f(x)>$ is represented by $g(x)\in\mathbb{F}[x]$ with
    $deg g < deg f$. Then the $gcd(f(x), g(x))$ (defined to be the largest degree monic
    common divisor) must be 1, since $f(x)$ is irreducible. So when we run the EEA, we
    obtain a solution for $a,b$ in $a(x)f(x) + b(x)g(x) = 1$. So in modulo $f(x) ,
    b(x) = g(x)^{-1}$
\end{proof}

\begin{eg}
    $f(x) = x^3 + 2x + 1\in\mathbb{Z}_3[x], g(x) = x^2 + 1$. Exercise check
    that $f(x)$ is irreducible. Now find $g(x)^{-1}$.
    \begin{align*}
        (x^3+2x+ 1) = x(x^2+1) + (x+1)\\
        (x^2 + 1) = (x+2)(x+1) + 2\\
        2 = (x^2+1) - (x+2)(x+1)\\
        = (x^2+1) - (x+2)((x^3+2x+ 1) - x(x^2+1))\\
        2 = (x^2 + 2x+ 1)g(x) + (2x+1)f(x)\\
        1 = (2x^2 + x  + 2)g(x) + (x+2)f(x)\\
        g(x)^{-1} = 2x^2 + x + 2
    \end{align*}
\end{eg}

If $f(x)\in\mathbb{Z}_p[x]$ is irreducible, then what is the order of $\mathbb{F} =
\mathbb{Z}_p[x]\setdiff<f(x)>$

Answer: In fact, if $deg f = n$, then $1,x,x^2,..x^{n-1}$ gives a basis for
$\mathbb{F}$ as vector space over $\mathbb{Z}_p$, so $|\mathbb{F}| = p^n$

Question: Given $p$ and $n$ is there $\mathbb{F}$ and $|\mathbb{F}| = p^n$

Answer: Yes, in fact, there exists $f(x) \in\mathbb{Z}_p[x]$ which is irreducible
of $deg n$, for any $n\geq 1$ so $\mathbb{F} = \mathbb{Z}_p[x]\setdiff<f(x)>$ works.
In fact
\begin{thm}
    For any prime $p$ and $n\geq 1$, there is a unique field of oder $p^n$ up to
    isomorphism.
\end{thm}

\begin{defn}
    Given a finite field $\mathbb{F}$ of characteristic $p$, and order $p^n$, the
    Forbenius map is the map $\pi:\mathbb{F}\rightarrow\mathbb{F}, x\mapsto x^{p^n}$
\end{defn}

This is a sort of "conjugation", and has anologis with complex conjugation.

\begin{thm}
    If $|\mathbb{F}| = p^n$, then every element of $\mathbb{F}$ is fixed by
    Frobenius map so $x^{p^n} = x, \forall x\in\mathbb{F}$.
\end{thm}

Proof sketch: Exercise! Look up the proof of Fermat's theorem.

\begin{eg}
    In $\mathbb{Z}_p$, $x^p = x \forall x\in\mathbb{Z}_p$. This is FlT.
\end{eg}

\begin{eg}
    In $\mathbb{F} = \mathbb{Z}_p[x]\setdiff<f(x)>$ with the $f$ monic irreducible of
    $deg n$. we have $x^{p^n} = y \forall y\in\mathbb{F}$.
\end{eg}

Question: how do you check if $y\in\mathbb{Z}_p$?
Answer: $y\in\mathbb{Z}_p \subset\mathbb{Z}$ iff $y^p = y$

Why? We know that each $y\in\mathbb{Z}_p$ satisfies this. Futhermore, this gives
$p$ roots of $y^p-y\in\mathbb{F}[y]$, and so that is in fact of all of them because
a $deg p$ polynomial has at roots.

\begin{thm}[Freshman's Dream]
    In a field $\mathbb{F}$ of characteristic $p$, for $a,b\in\mathbb{F}$,
    $(a+b)^p = a^p + b^p$.
\end{thm}

\begin{proof}
    By the binomial theorem,
    \begin{align*}
        (a+b)^p = a^p + .... + b^p.
    \end{align*{}
    so for $1\leq k\leq p-1$, $(p k)$(binomial) $\equiv 0\bmod{p}$
    Since $\mathbb{Z}_p$ the denominaotr is non-zero but numerator is 0.
\end{proof}

\subsection{Finite Fields Summarizd}
A finite field of order $q$ exits iff $q$ is a prime power: $q = p^n$,
$p$ prime, $n\geq 1$. In fact, there is is a unique field up to
isomorphism and this denoted $\mathbb{F}_q$ or $GF(q)$.
$\mathbb{F}_q$ with $q=p^n$ is constructed as $\mathbb{Z}[x]\setdiff <f(x)>$
for $f\in\mathbb{Z}[x]$ irreducible of degree $n$.

\subsection{Linear Code}
Now we use finite fields to define code with structure. This will mean
easier analysis and encoding/decoding.

\begin{defn}
    A linear $(n, k)$-code over $\mathbb{F}$ for $\mathbb{F}$ a finite
    field is a vector subspace $S$ of $\mathbb{F}^n$ with $dim(S)=K$.
    Note: we will think of $\mathbb{F}^n$ as consisting of row vectors.
\end{defn}

\begin{defn}
    A subsapce $S$ of $\mathbb{F}^n$ is a non-empty subset such that
    \begin{enumerate}
        \item $\forall a,b\in S, a+b\in S$ (closure under $f$);
        \item $\forall a\in S, \lambda a\in S$ (closure under scalar multiplication)
    \end{enumerate}
\end{defn}

\begin{eg}
    $C = \{000, 111\}$ is a linear $(3,1)$-code over $\mathbb{F}_2$. Check:
    closed under +, and *. $(111)$ is a basis for that subspace.
\end{eg}

\begin{eg}
    $C = \{010, 101\}$ is not a linear code. For many reasons... not closed under
    + $010 + 101 = 111 \not\in C$, etc.
\end{eg}

How many codewords in $(n,k)$-code over $\mathbb{F}_q$?

If $S\subset \mathbb{F}^n$ of dimension $k$, it has a basis
$v_1, ..., v_k$ so
$S = \{\lambda_1v_1 + \lambda_2v_2+ \lambda_kv_k: \lambda_1,...,\lambda_k\in
\mathbb{F}_q\}$. and thus $|S| = q^k$, since there are $q$ possibilities. for
each $\lambda$

Notation: A $(n,k)$-code over $\mathbb{F}_q$ is a $[n,q^k]$-code with alphabet
$\mathbb{F}_q$. Don't confuse these notations!!!

Fact: The information rate of an $(n,k)$-code $S$ is
$R = \frac{\log_q M}{n} = \frac{K}{n}$. We interpret a codeword in $S$ as
encoding the $k$ symbols. $(\lambda_1, ..., \lambda_k)$ via the $n$ symbols
in $\lambda_1v_1 + ... + \lambda_kv_k$

\begin{eg}
    $C = \{00000, 11100, 00111, 11011\}$ is a linear $(5,2)$-code over
    $\mathbb{F}_2$ . It has basis 11100, 00111, and we encode two bits,
    $b_1b_2$ as $b_1(11100) + b_2(00111)$, e.g.
    $11\mapsto 1(11100) + 1(00111) = 11011$
\end{eg}

Note: This encoding function depends on choice of (ordered) basis!
\subsection{Generator Matrices}
Given a linear $(n,k)$-code $S$ over $\mathbb{F}_q$ and with an
ordered basis $v_1, ..., v_k$ we form a generator matrix
$G = [v_1, ..., v_k]$, and the message $m\in\mathbb{F}_{q}^k$ is
encoded as $m\mapsto c = mG$

\begin{eg}
    For $C = \{00000, 11100, 00111, 11011\}$,
    $G = [1 1 1 0 0] [0 0 1 1 1]$ is generator matrix, and we
    encode $m = (b_1, b_2)$
\end{eg}

\begin{defn}
    A generator matrix $G$ is in standard form if
    $G = [I_k | A]$ where $I_k$ is the identity matrix.
\end{defn}

\begin{eg}
    $[1 1 1 0 0][0 0 1 1 1]$ is not in standard forom. But, if
    we swap every second and the fourth symbols, we get a code
    $C' = \{00000, 10110, 01101, 11011\}$ with $G' = [1 0 1 1 0][0 1 1 0 1]$ being a
     standard form generator matrix for $C'$.
\end{eg}

\begin{defn}
    A systematic code is a linear code which has standard form generator matrix
\end{defn}

\begin{eg}
    $C$ is not a systematic code, but $C'$ is!
\end{eg}

\begin{thm}
    Every linear $(n, k)$-code can be transformed into a sytematic code by
    permutation of the coordinates.
\end{thm}

\begin{proof}
    Starting with any generator matrix $G = M^{k\times n}(\mathbb{F})$, our
    code $row(G)$ which is unchanged under elementary row operations.
    So by row reduction, we can assume that $G$ is in reduced row echelon
    form. So there coordinates $a_1, ..., a_k$ such that the corresponding
    columns of $G$ forms an identity submatrix. (these are the columns with
    leading 1's). So if we permute the columns such that $a_1, ..., a_k$
    become the first $k$ columns, we get a modified code with generator matrix
    $G'$ in standard form.
\end{proof}

\begin{eg}
    $C = \{00000, 11100, 00111, 11011\}$ is not systematic. But! that is
    effective, so let's illustrate its:
    $G = [[1 1 0 1 1], [1 1 1 0 0]]$ RREF, then permute to standard form.
    So $C' = \{00000, 11100, 01011, 10111\}$ is systematic
\end{eg}

Note; $C'$ has a nice encoding function!
$00\mapsto 00000, 01\mapsto 01011, 10\mapsto 10111, 11\mapsto 11100$

Fact: With the standard form $G$, the codeword contains the message in first $k$ symbols.
The last $n-k$ symbols are added redundancy that gives error correcting
detecting capabilities.

\begin{eg}
    A common type of 1-error detecting code is contained by adding a bit at the end
    of a binary word, such that there are an even \# of 1's, so
    $101111\mapsto 101111, 111100\mapsto 1111000$ This is a parity check-bit.
    This gives a $(n+1, n)$-code. This has distance 2 (exercise).
\end{eg}

\begin{eg}
    The parity check code $C = \{0000, 0011, 0101, 0110, 1001, 1010, 1100, 1111\}$
    has a nice description. $C = \{x\in\mathbb{F}^4_2: x*1111 = 0\}$, dot product.
\end{eg}

\begin{defn}
    The dot product of $(x_1, ..., x_n)$ and $(y_1,...,y_n)$ is
    $x_1y_1 + ... + x_ny_n \in \mathbbb{F}$. If 0, then orthogonal.
\end{defn}

\begin{eg}
    WE note that $C$ can also be described a
    $C = \{x\in\{0,1\}: x_1 + x_2 + x_3 + x_4 = 0\}
    = \{x\in\{0,1\}: x*1111\}$
\end{eg}

In fact, $C$ is the orthogonal complement of the 4-replicaiton code
$C^{\perp} = \{0000, 1111\}$

\subsection{Linear Code Duality}
Now let's properly define those concepts
\begin{defn}
    Given $x,y\in\mathbb{F}^n$, we define
    $x\dot y = x_1y_1 + ... + x_ny_n\in\mathbb{F}$ and we
    call this the standard inner product
\end{defn}

\begin{defn}
    $x, y$ are orthogonal iff $x\dot y = 0$
\end{defn}

In fact, by property of $\dot$, we only need to have orthogonality
with basis vectors!

Lemma: $G$ is generation matrix for C, then $C^\{\perp} =
\{x*v_1=...=x*v_k=0\}$ code-word

\begin{proof}
    One diretion: is clear is since each $v_i$ is a codeword.
    Other direction: Suppose $x\in\mathbb{F}^n$ statisfies
    $x*v_i = 0$ for $i = 0, ..., k$
\end{proof}

If $G$ is the generation matrix, then the orthogonal space is
the nullspace of $null(G)$ (nullspace).

Then $C^{\perp} = Null(G)^T$. Transposed to get "row" vectors

\begin{thm}
    If $C$ has generator matrix, $G = [I_k | A]$, then $C^{\perp}$
    has generator matrix $H = [-A^T\mid I_{n-k}]$
\end{thm}

Note: Since all linear codes are equivalent to systematic codes, we
can extend this via permutation of coodinates.

\begin{cor}
    if $C$ is a $(n,k)$-code, then $C^{\perp}$ is a
    $(n, n-k)$-code.
\end{cor}

Note: If $G$ is any generation matrix for $C$ and
$H\in M^{n-k}\times n(\mathbb{F})$ of full rank satisfies
$GH^T$, then $H$ is a generation matrix for $C^T$

\section{Distance of Linear Codes}
lemma: $d(C) = min \{\text{Hamming weight of} c\in C\}$

\section{February 5, 2018}
\begin{defn}
    The Hamming distance $w(v)$ of $v\in\mathbb{F}^n$ is the number of non-zero
    coordinates in $V$.
\end{defn}

Note: $w(v) = d(v, 0)$ is the Hamming distance to 0.

\begin{defn}
    The Hamming weight of a code $C$ is $w(C) = min\{w(C): c\in C\setdiff\{0\}\}$.
\end{defn}

\begin{thm}
    If $C$ is a linear code, then $w(C) = d(C)$.
\end{thm}

\begin{proof}
    We compute $d(C) = min\{d(x,y): x,y\in C, x\ne y\}$,
    \begin{align*}
        d(C) = min\{w(x-y): x, y\ne y\}\\
        d(C) = min\{w(c): c\in \setdiff \{0\}\}\\
    \end{align*}
\end{proof}

Note: We can represent this with our sphere packing analogy.

\begin{eg}
    If $C$ has the following generator matrix over $\mathbb{F}_3$:
    \[
        G =
        \begin{bmatrix}
            2 & 0 & 2 & 1 & 0\\
            1 & 1 & 0 & 0 & 1\\
        \end{bmatrix}
    \]
    There are $3^2 = 9$ codewords.
\end{eg}
To get $C^{\perp}$, simply apply the formula. Now we note $c\in C$ satisfies
$Hc^T = 0$, so we get a linear dependency among the columns of $H$. For example,
$c = 11001\in C$ gives a linear dependency.

So, we note that every codeword from $C$ corresponds to a linear dependency amoung the
columns of $H$, hence $d(c)$ is the minimum number of linearly dependent columns in $H$.

Lemma: Given a linear code $C$ with parity-code matrix $H$, every codeword in $C$
corresponds to a linear dependency among columns of $H$, and vice versa.

\begin{thm}
    Let $H$ be a parity-check matrix for $C$, then $d(C)\geq S$ iff every collection of
    $s-1$ columns of $H$ are linearly dependent over $\mathbb{F}$
\end{thm}

\begin{proof}
    Suppose that there is a collection of $s-1$ linearly dependent columns of $H$.
    For simplicity of notation, take them to be the first $s-1$ columns so if
    $H = [h_1 h_2 ... h_{s_1} h_s ... h_n]$ is $H$ split into columns. Then there are
    $\lambda_1, \lambda_2, ..., \lambda_{s-1}\in\mathbb{F}$, not all 0, such that
    $\lambda_1h_1 + \lambda_2h_2 + ... + \lambda_{s-1}h_{s-1} = 0$.
    Then, $c = (\lambda_1, \lamda_2, ..., \lambda_{s-1}, 0, 0 ..., 0)\in\mathbb{F}^n$
    satisfies $Hc^{T} = 0$ so $c\in (C^{\perp})^{\perp}$, but clearly
    $w(c)\leq s-1$ and $c\ne 0$, so in fact, $d(C) \leq s-1$. For other implication,
    suppose that $d(C)\leq s-1$, then there is a codeword $c\in C$ with $w(c)\leq s-1$

    For simplicity of notation, assume that all non-zero entries of $c$ appear in the
    first $s-1$ coordinates, so $c = (\lambda_1, \lambda_2, ..., \lambda_{s-1}, 0, ..., 0)
    \in\mathbb{F}^n$, and not all $\lambda_i$ are 0. Then $Hc^T = 0$, so we have a linear
    dependency among the first $s-1$ columns of $H$
    $\lambda h_1 + ... + \lambda_{s-1}h_{s-1} = 0$
\end{proof}

What is the distance of a code?\\
Method 1: Check all 16 codewords

We note that columns dependent in $H$, with 1, 2, 4, then $d(C)  \leq 3$. On the other
hand, no single column is dependant, since none of them is 0. Hence $d(C) > 1$

Now, let's consider a set of two columns of $H$ over $\mathbb{F}_2$, these are linearly
dependent iff there are two identical columns. This is not the case, so $d(C) > 2$

\begin{lemma}
    A binary code with PCM has
    \begin{enumerate}
        \item $d(C) = 1$ iff $H$ has a 0 column
        \item $d(C) = 2$ iff $H$ has no 0 column, and two identical columns.
        \item $d(C) \geq 3$ if 1 wand 2 don't hold.
    \end{enumerate}
\end{lemma}

The code is called the Binary Hamming Code of order 3 and has distance 3.

\subsection{Hamming Code}
...
\begin{thm}
    The $n$-replication code for $n$ odd is perfect (Binary)
\end{thm}

\begin{lemma}
    A code is perfect iff every word $A^n$ is at distance $\leq e$
    from a codeword.
\end{lemma}

\begin{thm}
    The Hamming Codes are perfect
\end{thm}

\begin{thm}
    The only perfect linear codes (up to equivalence) are
    \begin{enumerate}
        \item Trivial codes: $\mathbb{F}^n$ is an $(n,n)$-code
        \item Binary replication codes of odd length
        \item The Hamming Codes
        \item The $(23, 12, 7)$-binary Golay code
        \item The $(11,6,5)$-ternary Golay code
    \end{enumerate}
\end{thm}

\subsection{Decoding Linear Codes}
This is hard in general. We will see after the break after studying Group
Theory.

Single errors are manageable. Suppose $C$ is an $(n,k)$-code over $\mathbb{F}$, with
PCM (parity code matrix) $H$

We will take advantage of linear algebra! Starting point: We note that
$c\in C$ iff $Hc^T = 0$.

What if we perturb $c$?\\
$H(c+e)^T = Hc^T + He^T = 0 + He^T = He^T$. So if
$w(e) = 1$, then we can see that $H(c+e)^T = He^T$ is going to be a multiple of
a column of $H$!

\begin{defn}
    If $c\in C$ is sent and $r\in\mathbb{F}^n$ is received, then the error vector
    is $e = r-c$ (so $c+e = r$)
\end{defn}

Algorithm to decode single errors:
\begin{enumerate}
    \item If $r$ is received, comput $s = Hr^T$
    \item If $s = 0$ , accept $r$
    \item If $s \ne 0$, compute $s$ to each column of $H$, and determine if
        $s = \alpha h_i$ for some column $h_i$ and scalar $\alpha$.
    \item Decode to to $c = r - e$ where $e = (0, ..., \alpha, 0, ...)$
    \item Reject otherwise. This is an implementation of IMLD.
\end{enumerate}

Questions: These can cover anything I wrote on board, including exercises.
\begin{enumerate}
    \item Basics: Multiple short questions (25pts)
        \begin{enumerate}
            \item Specific codes we've defined
            \item Basic terminology/notation
            \item Information rate, Hamming distance, and properties
            \item Channel encoding/decoding, big picture
        \end{enumerate}
    \item Coding Theory! (Error Correcting) (15 pts)
        \begin{enumerate}
            \item Maybe: Design code given specs
            \item Maybe: Design a decoder
            \item Maybe: Prove an important result, or partial result
        \end{enumerate}
    \item Finite Fields (15 pts) Assignment
    \item Linear Codes, up until now (25 pts)
\end{enumerate}

Let $C$ be an $(n, k)$-code over $\mathbb{F}$. Recall that $\mathbb{F}^n$
is an Abelian Group of order $q^n$, and $C$ is a subgroup of order $q^k$.
So we can partition $\mathbb{F}^n$ into $C_0 = C_1, C_2, ..., C_{t-1}$ with
$t = \frac{q^n}{q^k} = q^{n-k}$.

Crucial observation for decoding:

If $c\in C$ is transmitted, and $r$ is received, then the error vector
$e = r-c$ satisfied $r\equiv e\bmod{C}$. In other words, $r$ and $e$ are
in the same coset.

\begin{eg}
    Let $C$ be the binary $(4,2)$-code with GM. $G = [[ 1 1 0 0 ], [ 0 0 1 1 ]]$
    \begin{align*}
        C_0 = C = \{0000, 1100, 0011, 1111\}\\
        C_1 = C + 1000 = \{1000, 0100, 1011, 0111\}\\
        ...
        C_3 = C + 1001 = \{1001, 0101, 1010, 0110\}\\
    \end{align*}
\end{eg}

Either 0000 was send and $e = 1010$\\
Either 1100 was send and $e = 0110$\\
Either 0011 was send and $e = 1001$\\
Either 0101 was send and $e = 0101$\\

This confirms our observation that $e$ must be in the cost of $r = 1010$, which is
$C_3$.

Question: What would IMLD do? Reject! Since all codewords are at distance 2, and
ties are rejected.

what if $r = 0100$? We can use the table from earlier to set up a CMLD decoder.
For each coset, we choose a coset leader to be an element of lowest weight. So
for, $C_0$, it's 0000, for $C_1$ it's 1000, $C_2$ it's 0010, and $C_3$ it's
1001. Then our CMLD always assumes that for a received codeword $r\in C_i$ , the
error was the coset leader. So $c=0100$ has $e=1000$ and is decoded to $c=1100$.
By the way, we wrote down the table, there is a simple description for our CMLD.
\begin{enumerate}
    \item Find $r$ in the table
    \item More to top of table, and out $c$ this is standard array deoding.
\end{enumerate}

Algorithm: Std Array Decoding

Compute a std array with rows $c_0, c_1, ..., c_{t-1}$, and entry in row $i$, and
column $j$ obtained by adding the coset leader $l_i$ of each $C_i$ to the codeword
$c_j\in C$ at the top of its column. $l_i$ must have min weight.

Input: A word $r$

Step:
\begin{enumerate}
    \item Locate $r$ in std array, say $r\in C_i$
    \item Correct $r$ to $c = r - l_i$ at the top of the column
\end{enumerate}

Observation If $r = c+e$, then we multiply through by $n$, PCM H:
\begin{align*}
    Hr^T = Hc^T + He^T\\
    = 0 + He^T = He^T\\
\end{align*}

In other words, for each element of a coset $c_i$, we get the same
result when we multiply by $H$.
$x,y$ are in same coset $\iff$ Hx^T = Hy^T.
This allows us to summarize the std array.

Modified algorithms: Give $r$, calculate $Hr^T$, locate in new table,
assume $e$ is coset leader decode to $c=r-e$.

For example, if $r = 10001$, then $(Hr^T)^T = 110$, so we assume
$e = 01000$, and decode to $c = r-e = 11001$.

Note after precomputing all coset leaders, and syndromes $Hr^T$,
this new algorithm saves space. Good for 60's applications.

Syndrome Decoding(CMLD)

Note: This is equivalent to std array decoding in terms of input/output. For
storage, we've reduced from for std array: $n*2^n$, for syndrome decoding:
$n2^{n-k}$, $n$ bits for each coset leader.

\begin{defn}
    The Binary Golay Code is the (23, 12)-binary code with generator matrix
    $[I_{12} | \hat{B}]_{12\times 23}$
\end{defn}
Properties: $d(c_{23}) = 7$ (proof later). This is a perfect code.
Check:
\begin{align*}
    |c_{23}|\sum_{i=0}^e \binom{n}{i} = 2^{23}
\end{align*}

To study the mathematical properties, it is useful to study the
related code. The Extended Binary Golay Code is a (24, 12)-code
with $d = 8$.
\begin{defn}
    Let $B = [0 1 ... 1]^T\hat{B}$ (noting $B^T = B$)
\end{defn}

Then the generator matrix for Extended Golay Code is $G = [I_{23} | B]$.
notated $C_{24}$.

Properties of $c_{24}$
\begin{enumerate}
    \item Used on 1979 Voyager Space mission to Jupiter + Saturn
    \item $GG^T = 0$
\end{enumerate}

$H = [-B^T | I_{12}] = [B | I_{12}]$ is a pcm.
Hence $H = [B | I_{12}]$ is also a generator matrix for $C_{24}$ which come
in handy.

$d(C_{24}) = 8$
\begin{proof}
    Note that each row vector $r_i$ has weight 8 or 12. $d(C_{24}) = 8$. Note that
    $4\mid w(r_i)$ for any two rows $r_i, r_j$, and
    $r_i \dot r_j = 0$. Hence there are an even $#$ of
    positions with both having bit 1. As a result, $r_i + r_j$ also has
    $4\mid w(r_i + r_j)$. Since $#1$'s in first row is multiple of 4, and
    also for 2nd row, to obtain third row, we comibine those , and strike
    out an even number of pairs of 1s. Hence, in third row, there is $#$ of
    1s that is a multiple of 4. So $4\mid w(r_i+r_j)$. Similarly by induction,
    we seee that $4\mid w(r_i + r_j + r_k)$ etc. This idea gives a proof that
    $4\mid w(c) \forall c\in C_{24}$ As a rseult, $w(C)$ is either 4 or 8. so
    to show $d(C) = 8$ we only need to show that no codeword has weight 4.
    We consider cases:
    \begin{itemize}
        \item No single row has weight 4
        \item Consider $r_i + r_j$ and we can check that $w(r_i + r_j) = 8$.
    \end{itemize}
    Consider $w(r_i + r_j + r_k)$. If $w(r_i + r_j + r_k) = 4$, consider the
    first and second half of the the bits $r_i + r_j + r_k = x\mid y$. We
    have $w(x) = 3$, so $w(y) = 1$. Trick: consider now $H = [B | I_{12}]$ which
    is also a generator matrix. Since $w(y) = 1$, in fact $x\mid y$ must be
    a row of H! But these all have weight 8 or 12.... Contradiction. Hence
    $wt(r_i + r_j + r_k) \geq 8$

    Consider $w(r_i + r_j + r_k + r_l) = 4$. Similarly we have $w(x) = 4,
    w(y) = 0$, so $y = 0$, Looking at $H$, that means that in factor $x=0$
    Contradiction.

    What if we add $S$ or more together? Then $wt(x) \geq 5$, so we can't get
    a weight 4 codeword.
\end{proof}

\subsection{Decoding Algorithm for $C_{24}$}
For we compute a syndrome for a received word $r = (x,y)$, where $x,y$ are the
first 12 bits and last 12 bits respectively. Let $e = (e_1, e_2)$ be the error
vector, split into halves. First, let's consider the syndrome
\begin{align*}
    S_1 = Gr^T = [I_{12} | B](x,y)^T\\
    = I_{12}x^T + By^T = x^T + By^T
\end{align*}

Noting that $G$ is a PCM since $C_{24}$ is self-dual.

So we seek a weight 3 or less $e = (e_1, e_2)$ such that
$s_1 = e^T + Be_2^T$, so we can correct to $c = r - e$.

(Recall that this is how the standard array/syndrome decoder functions)
Let's consider cases:
\begin{enumerate}
    \item If $s_1 = 0$, then $r\in C_{24}$ so accept $r$
    \item If $w(s_1) \leq 3$, then we assume that $e_1 = s_2, e_2 = 0$, then
        indeed $Ge^T = s_1^T + B0^T = s_1^T$
    \item Consider the case where $w(e_1)\leq 2, w(e_2) = 1$, then we have
        $s_1^T = e_1^T + Be_2^T$, so if $s_1^T$ differs from a column of $H$
        in 0, 1 or 2 positions, we take $e_1$ to indicate those, and $e_2$ to
        to indicate the column.
\end{enumerate}

\begin{eg}
    Decode $r = (1000 1000 0000 1001 0001 1101)$. We compute
    $s_1^T = x^T + By^T = (0100 1000 0000)^T$. We note that that
    taking $e_1 = s_1, e_2 =0$, we get the same syndrome:
    $e_1^T + Be_2^T = s_1^T$
    So take
    $e = (0100 1000 0000 0000 0000 0000)$, and we decide to
    $c = re = (1100 0000 0000 1001 0001 1101)$.
    (exercise: check that $Gc^T = 0$).
\end{eg}

\begin{eg}
    Decode $r = (1000 0010 0000 1000 1101 0010)$, We compute
    $s_1 = Gr^T = x^T + By^T = (1011 1110 1011)^T$. We note that
    $s_1^T$ differs from $b_4$ (column 4)  in posiitons 6 and 7. So
    we assume $e_1 = (0000 0110 0000)$ and $e_2 = (0001 0000 0000)$
    and we correct to $c = r - e = (1000 0100 0000 1001 1101 0010)$
    So we have covered the cases where $w(e_2) \leq 1$, what if
    $w(e_2) = 2, 3$, (still with $w(e) \leq 3$ since $d(C_{24}) = 8$
\end{eg}

$r = (0100 0010 0000 0100 1101 1011)$
Compute $s_1 = Gr^T = x^T + By^T$, and we see that this has weight
$\geq 3$, and differs in $\geq 4$ positions from each column, so
this excludes the possibility that $w(e_2) \leq 1$. So instead, we
consider the other cases, still assumming $w(e) \leq 3$, so in
cases $w(e_1)\leq 1$, since $w(e_2)\geq 2$, so instead compute the
syndrome using $H = [B|I_{12}]$,
$s_2^T = Hr^T = Bx^T + y^T$, so we see that we cannot assume $e_1 = 0$,
or else $w(e_2) = 5$ is too big, we compare this to columns (or rows) of
$B$, and see that this differs from $b_2$ in positions 1 and 5.


Input: $r = (x,y)$ ($x,y$ with 12 bits)
\begin{enumerate}
    \item Compute $s_1 = x^T + By^T$. If $s_1 = 0$, output $r$, stop.
    \item If $w(s_1) \leq 3$, decode to $(x-s_1, y)$ stop
    \item If $d(s_1, b_i)\leq 2$ for some $i\in\{1,2,...,12\}$, then,
        let $e_2 = (00000,1,000000)$ indicate positon $i$, and let
        $e_1$ indicate the 0, 1, 2 bits that flipped then decode to $r-e$.
        Stop
    \item Computer $s_2^T = Bx^T + y^T$ (note: $s_2 \ne 0$ or else step 1, stop)
    \item If $w(s_2)\leq 3$, assume $e_1 = 0, e_2 = s_2$, decode to $r-e$, stop
    \item IF $d(s_2, b_i) \leq 2$ for some $i\in\{1,2,...,12\}$, let $e_1$ denote
        position $i$ with a 1, and let $e_2$ denote the $\leq 2$ flipped bits,
        Decode to $r-e$, stop.
    \item Reject $r$!
\end{enumerate}

Question: Suppose $d(s_1, b_i) = 3$. What shortcut could our decoder take?
Say $s_1 = (0000 1111 1111) = e^T_1 + Be_2^T$. Then we might have
$e_2 = (1000 0000 0000), e_1 = (0111 0000 0000)$, then $c=r-e$ is a codeword
with $d(c,r) = 4$. Then we note that $d(c', r) > 3$ for any $c'\in C$ by triangle
inequality, so reject!

\subsection{March 7, 2018}
Golay Code: Recall that $C_{24}$ with $G = [I_{12}| B], H = [B|I_{12}]$,
$(24, 12)$-binary code with $d(C_{24}) = 8$. If we delete first column of $B$ to
get $\hat{B}$, we get a generator matrix $\hat{G} = [I_{12}|\hat{B}]$ for a
$(23, 12)$-code $C_{23}$ with $d(C_{23})$, with $d(C_{23}) 7$. This is the
Golay Code, $C_{23}$ is a perfect code!
check: sphere packing argument.

$C_{23}$ has another property, if we change the bit order a appropriately to get
$\tilda{C}_{23}$. This is a cyclic code.

\subsection{Cyclic code}
\begin{defn}
    A subspace $S$ of $\mathbb{F}^n$ is called a cyclic subspace if
    $(a_0, a_1, ..., a_{n-1})\in S, \implies (a_{n-1}, a_0, a_1,...,a_{n-1})\in S$.
    A linear code $C$ is a cyclic code if $C$ is if $C$ is a cyclic subspace.
\end{defn}

\begin{eg}
    $C = \{000, 110, 011, 101\}$ is a cyclic code. Cyclic codes have a lot of
    algebraic structure. To see this, we go back to polynomial rings. Let
    $R = \mathbb{F}[x]\setdiff\langle x^n-1\rangle$ for field $\mathbb{F}$. This
    is a commutative ring. (Not a field if $n > 1$)
\end{eg}

First we view $R$ as a vector space over $\mathbb{F}$ with basis
$\{1, x, x^2, ..., x^{n-1}\}$. since $R = \{[a_0 + a_1x + ... + a_{n-1}x^{n-1}]\}$.
(by division algorithm). In this way, we will identify $\mathbb{F}^n$ with $R$.
Explicitly, $(a_0, a_1, ..., a_{n-1}) \in \mathbb{F}^n, \leftrightarrow
a_0 + a_1x + ... + a_{n-1}x^{n-1} \in R$.

\begin{eg}
    $C = \{000, 110, 011, 101\}$ corresponds to
    $\{0, 1+x, x+x^2, 1+x^n\}\subseteq \mathbb{F}_2[x]\setdiff x^3-1$
\end{eg}

Why modulus $x^3 -1$? Note that multiplication by $x$ corresponds to cyclic shift
\begin{align*}
    x(1+x) = x + x^2\\
    x(x+x^2) = x^2+x^3 = 1 + x^2\\
\end{align*}

In general: The cyclic shift operation in $\mathbb{F}^n$ coresponds to
multiplication by $x$ in $R = \mathbb{F}[x]/\langle x^n-1 \rangle$

\begin{defn}
    A non-empty subset $I$ of $R$ [or any commutative ring] is an
    ideal of $R$ if
    \begin{enumerate}
        \item $(I, +)$ is a group
        \item $\forall a\in I, b\in R, a\cdots b\in I$
    \end{enumerate}
\end{defn}

\begin{thm}
    Let $S$ be a non-empty subset of $\mathbb{F}^n$. Let $I$
    be the corresponding polynomials in $R$. Then $S$ is a cyclic
    subspace of $\mathbb{F}^n$ if and only if $I$ is an ideal of $R$.
\end{thm}

This theorem give us access to a lot of tools to study cyclic codes.

\begin{thm}
    Let $S$ be a non-empty subset of $\mathbb{F}^n$. Let $I$ be the
    corresponding polynomials in $R$. Then $S$ is a cyclic subspace of $\mathbb{F}^n$
    if and only if $I$ is an ideal of $R$. This theorem will give us access to a
    lot of tools to study cyclic codes.
\end{thm}

\begin{proof}
    Suppose $S$ is a cyclic subspace of $\mathbb{F}^n$. Then $S$ is closed under
    addition, so $I$ is as well. $I$ is also closed under scalar multiplication.
    Now let $a(x) \subset I$, and $b(x) = b_0 + b_1x + ... + b_{n-1}x^{n-1}\in R$.
    Since $S$ is a cyclic subspace, $xa(x)\in I$, so therefore,
    $x(x(x)) = x^2a(x)\in I$, etc. $x^ka(x)\in I$. Hence, by closure of +,
    we will also have $b_0a(x) + b_1xa(x) + ... + b_{n-1}x^{n-1}a(x)\in I$.
    so $(b_0, b_1x+ ... + b_{n-1}x^{n-1})a(x) = b(x)a(x) \in I$. We are done.

    Suppose $I$ is an ideal of $R$. Then since $I$ is closed under addition and
    multiplicaiton by elements of $\mathbb{F}$, it follows that $S$ multiplicaiton
    by elements of $\mathbb{F}$, it follows that $S$ is a vector subspace of
    $\mathbb{F}^n$. Finally, since $I$ is closed under multiplication by $x$,
    $S$ is closed under cyclic shifts, so $S$ is a cyclic subspace.
\end{proof}

\begin{defn}
    Let $R$ be any commutative ring, and take $g\in R$, we define
    $\langle g \rangle = \{g*r: r\in R\}$, to be the principal ideal
    generated by $g$. (Any ideal $I\subset r$ if which such a $g$
    exists is called a principal ideal.)
\end{defn}

\begin{eg}
    $\langle x-1 \rangle \subset \mathbb{F}_3[x]\setdiff \langle x^3\rangle$
    is a principal ideal.
\end{eg}

Fact: $\mathbb{F}[x]\setdiff\langle x^n -1\rangle$ is a principal ideal ring
meaning that all ideals are principal

\begin{proof}
    Let $I$ be an ideal of $\mathbb{F}[x]\setdiff \langle x^n-1\rangle$.
    Case: If $I = \{0\}$, then $I = \langle 0\rangle$
    Case: If $I\ne \{0\}$, let $g(x)\in 1$ be a lowest degree polynomial,
    with $g(x) \ne 0$. Then take any $h(x)\in I$. Then divide by $g(x)$
    $h(x) = q(x)g(x) + r(x), deg r < deg h$, then in $R$, $h(x)-q(x)g(x)\in I$
    (since $g(x), h(x) \in I$, and $I$ is closed under + and multiplication
    by $(-q(x))$, so $r(x) = h(x) - q(x)g(x) \in I$, but $deg r < deg g$!, so
    in fact $r = 0$, to avoid a contradiction since $deg g$ is minimal.
\end{proof}

hence $g(x)\mid h(x)$. It follows that $I = \langle g(x) \rangle$

\begin{defn}
    $g(x)\in\mathbb{F}[x]$ is monic if the leading coefficient is 1.
\end{defn}

Fact; Every non-zero ideal of $R = \mathbb{F}[x]\setdiff \langle x^n-1\rangle$
is of the form of $\langle g(x)\rangle$ for a monic $g(x)$, which is a min
degree non-zero element.

Analogy: $\mathbb{Z}$ is also principal ideal ring. Any ideal $I \subseteq\mathbb{Z}$
is of the form $I = \langle n \rangle = \{nk: k\in\mathbb{Z}\}$.

Question: Let $I$ be the ideal $I = \{12x + 15y: x,y\in\mathbb{Z}\} = \langle 12, 15\rangle$

By Bezout's lemma, $I = \langle 3 \rangle$ where $3 = gcd(12, 15)$.

\section{March 12, 2018}
\begin{thm}
    Let $I$ be a non-zero ideal of $R$.
    \begin{enumerate}
        \item There is a unique monic polynomical $g(x)$ of smallest degree in $I$, and
            $I = \langle g(x) \rangle$
        \item $g(x)\mid x^n -1$ in $\mathbb{F}[x]$
    \end{enumerate}
\end{thm}


\begin{proof}
    \begin{enumerate}
        \item Let $g(x), h(x)$ be both monic generators of lowest degree in $I$, so
            $g(x) -h(x)\in I$, also, and we see $deg(g-h) < deg g = degh$
            (monic), but $g$ has min degree in $I$, so we must have $g-h=0$
        \item Write $x^n - 1 = g(x)q(x) + r(x)$ by division algorithm with
            $deg(r) < deg(g)$, then $r(x) = -q(x)g(x)\in I = \langle g(x) \rangle$.
            But $deg g$ is minial in $I$, so we must have $r=0$. Hence $g(x) \mid x^n-1$
    \end{enumerate}
\end{proof}

\begin{eg}
    In $\mathbb{F}\setdiff \langle x^3-1 \rangle$, what is the generator polynomial
    of $I = \langle x^2 - x\rangle = \{a(x)(x^2 - x)\in R\}$?

    Note: by theorem, $x^2-x$ is not the generator polynomial since $x^2 - x\not\mid
    x^3 - 1$. So what is the monic generator polynomial?

    Let's try some stuff out:
    $x^2-x\in I$, so $x(x^2-x) = x^3 + x^2 = 1+x^2\in R$ (calculation in $R$).
    $x(1+x^2)\in I$, so $x+x^3 = x+1\in I$. This has smaller degree than
    $x^2-x$. In fact, $I = \langle x + 1\rangle$
\end{eg}

\begin{thm}
    If $g(x)$ is a monic divisor of $x^n-1$, then $h(x)$ is the (monic) generator
    polynomial of $I = \langle h(x) \rangle$
\end{thm}

\begin{proof}
    So how we've actually classified all cyclic codes!
\end{proof}

\begin{eg}
    What are all cyclic subspaces of $\mathbb{F}_2^3$? Solution:
    These correspond exactly to monic divisors of $x^3 -1\in\mathbb{F}_2[x]$.
    Irreducible factorization! $x^3-1 = (1+x)(1+x+x^2)$. So cyclic codes are
    $g_1(x) = 1: \langle g_1(x) \rangle = \mathbb{F}_2^3 =
    \{000, 001, 010, 011, 100, 101, 110, 111\}$\\
    $g_2(x) = 1+x: \langle g_2(x)\rangle = \{0, 1+x, x+x^2, 1+x^2\}$
    corresponding to $S = \{000, 110, 011, 101\}$
    $g_3(x) = 1+x+x^2: s_3 = \{000, 111\}$\\
    $g_4(x) = (1+x)(1+x+x^2): s_4 = \{000\}$\\
    (since $g_4(x) = x^3 - 1 = 0$ in $R$.)
\end{eg}

In general, $x^n -1 \in\mathbb{F}[x]$ has an irreducible factorization
$x^n - 1 = p_1(x)^{e_1}p_2(x)^{e_2}...p_k(x)^{e_k}$
into monic irreducible polynamial which are pairwise distince. So there are
many cyclic code of length n.
$(e_1 + 1)(e_2 + 1)....(e_k + 1)$

\begin{thm}
    Let $g(x)$ be a monic divisor of $x^n - 1$ with $deg(g) = n-k$. Then
    $I = \langle g(x) \rangle = \{a(x)g(x): a(x)\in R\}
    = \{\bar{a}(x)g(x) \in R: deg \bar{a} < k\}$
\end{thm}

We can represent any element of $I$ as $\bar{a}(x)g(x)$ with
$deg \bar{a} < k$, and $deg g = n-k$.
Futhermore, the $\bar{a}$ givves a unique way to represent any element
of $I$

\begin{proof}
    Take any element $h(x) = a(x)g(x) \in I$. So in $\mathbb{F}[x]$,
    $a(x)g(x) = q(x)(x^n -1) + h(x)$. But $g(x) \mid a(x)g(x)$ and
    $g(x) \mid x^n -1, g(x) \mid a(x)g(x) - q(x)(x^n -1) = h(x)$
    Hence $h(x) = \bar{a}(x)g(x)$ in $\mathbb{F}[x]$ for some
    $\bar{a}(x)\in\mathbb[x]$, and since $deg h$ can be taken to be
    $< n$, we must have
    \begin{align*}
        deg(\bar{a}) = deg h - deg g < n - (n-k) = k
    \end{align*}
\end{proof}

\begin{cor}
    If $g(x) \mid x^n -1, deg g = n-k$, and $g$ monic, Then the cyclic
    code $\langle g(x) \rangle$ hsa dimension $k$
\end{cor}

\begin{proof}
    By lemma, we have $I = \{a_0 + a_1x + ... + a_{k-1}x^{k-1})g(x): a_0, ...
    a_{k-1}\in\mathbb{F}\}$. and each representation is unique.
    So $I$ has basis $\{g(x), xg(x), x^2g(x), ..., x^{k-1}g(x)\}$, so $dim I = k$
\end{proof}

\begin{eg}
    Construct a $(7,4)$-cycle code over $\mathbb{F}_7$. Say this corresponds to
    $\langle g(x) \rangle$, where $g(x)\mid x^7-1$. So that $k = 4$, we need
    $deg g = n - k = 7-4 =3$
\end{eg}

Then as a polynomials of $deg < n-k$, they are:
$x^0, x^1, ..., x^{n-k-1} \bmod{g(x)}$ (for $I$ block)\\
$x^{n-k}, x^{n-k+1}, ..., x^{n-1} \bmod{g(x)}$

The syndrome is then simply a matter of taking the remainder upon
division by $g(x)$

Note: This means that we only need to know $g(x)$ (and we forget $H$)

\subsection{Cyclic Burst Error}
Instead of focusing on error correcting/detecting capability in general,
we narrow our focus to errors that occur in "bursts" now.

Observation: If error occurred in $r_0$, say, then $e = (1, 0, ..., 0)$
so syndrome is $e(x) \bmod{g(x)} = 1$, more generally, if the errors all
occur in the first $t$ symbols (for $t\leq n-k$), then
$e(x) = (e_0, e_1, e_2, e_3,...,e_{t-1}, 0, ..., 0)$ will have syndrome
$e(x) \bmod{g(x)} = e_0 + e_1x + ... + e_{t-1}x^{t-1} = e(x)$
Point is: it is simple to decode errors that all occur at the beginning.

Observation2: If $r(x)$ has syndrome, $s(x)$, then it is simple to
compute the syndrome of its cyclic shfit:
\begin{thm}
    If $r(x)$ has syndrome $s(x) = s_0 + s_1x + ... + s_{n-k-1}x^{n-k-1}$,
    then the syndrome of $xr(x)$ is
    \begin{enumerate}
        \item $xs(x)$ if $s_{n-k-1} = 0$
        \item $xs9x) - s_{n-k-1}g(x)$ if $s_{n-k-1} \ne 0$
    \end{enumerate}
\end{thm}

\begin{proof}
    we have $r(x) = l(x)g(x) + s(x)$ for some $l(x)\in\mathbb{F}[x]$,
    so $xr(x) = xl(x)g(x) + xs(x)$
    \begin{enumerate}
        \item If $s_{n-k-1} = 0$ then $deg(xs(x)) < n-k = deg g$, so
            $xs(x)$ is the remainder of $xr(x)/g(x)$. So it the syndrome.
        \item If $s_{n-k-1} \ne 0$, then
            \begin{align*}
                xr(x) = xl(x)g(x) + xs(x) - s_{n-k-1}g(x) +s_{n-k-1}g(x)\\
                = (xl(x) + s_{n-k-1})g(x) + (xs(x) - s_{n-k-1}g(x)\\
                x(...+s_{n-k-1}x^{n-k-1}) - s_{n-k-1}(...+x^{n-k})
            \end{align*}
    \end{enumerate}
\end{proof}

\begin{defn}
    Let $e\in\mathbb{F}^n$. The cyclic burst lengt of $e$ is the
    shortest cyclic block in $e$ containing all non-zero elements of $e$\\
    $e$ is a cyclic burst error of length $t$ if its burst length is $t$.
\end{defn}

\begin{defn}
    $C$ is an t-cyclic burst error correcting code if all cyclic burst errors
    of length $\leq t$ are in different cosets of $C$. The lenght such $t$ is
    the cyclic burst error correcting capability of $C$.
\end{defn}

\begin{eg}
    $g(x) = 1+x+x^2+x^3+x^9$ generates a binary $(15,9)$-cyclic code $C$,
    Check $g(x) \mid x^15-1$. Let's use syndrome decoding:
    Let $E = \{\text{cyclic burst errors of length $\leq 3$}\}$, and we
    check that these all have different syndromes.
\end{eg}

Check: the integer representation of syndrome are all different. Thus we have 3-cyclic
burst error correcting code!

observation: This is one way to check burst error correcting capability. We can then use
table for syndrome decoding. But there is a better way! Suppose error occurred in the
middle $e = 0000 0001 1100 0000$, $r = c+e$

This idea leads to error trapping algorithm.

\section{Error Trapping}
This is a decoding algorithm for $C$.
Idea: Suppose $e$ is an error vector with cyclic burst length $\leq t$. Then some
cyclic shift of $e$, say $e_i \leftrightarrow x^ie(x)$ has all non-zero entries in
the first $t$ positions. Then the syndrome $s_i = He_i^T$ has burst length $\leq t$,
and we take $e_i = (s_i, 0)$. Lastly, we shift it all the way around,
$e(x) = x^{n-i}e_i(x)$ and correct $r$ to $c = r-e$.

Question: How do we compute $s_i$?
(Recall: the syndrome of $r(x)$ is $r(x)\bmod{g(x)}$. We start by computing
$s_0(x) = r(x) \bmod{g(x)}$. If $deg s_0 < t$, then we can take $e=(s_0, 0)$,
and decode to $r-e$.
Otherwise, we compute $s_i(x)$ using rule form last lecture:
\begin{align*}
    s_1(x) = xr(x)\bmod{g(x)}\\
    = xs_0(x)\bmod{g(x)}\\
    = xs_0 if $deg < n-k$
    = xs_0(x) - (leading coefficient $s_0$)g(x) otherwise.
\end{align*}

Algorithm: Error Trapping!
Input: $r(x)$ is received word
Steps: For i from 0 to i:
    compute $s_i$
    If $deg(s_i) < t$, then
    let $e(x) = x^{n-i}s_i(x)$
    correct $r(x)$ to $c() = r(x) - e(x)$

Recall:
$g(x) = 1 + x + x^2 + x^3 + x^6$ is a 3-cycle burst ecc.
Decode $r = (1110 1110 1100 000)$
Solution: $r(x) = 1+x+x^2+x^4+x^5+x^6+x^8+x^9$. Compute
$s(x)$ by long division.

$s(x) = 1+x+x^4+x^5$, $s_0 = (1100 1100 0000 000),
deg 3 \geq 3$, so we can't correct at the moment.
Next $s_1(x) = xs(x)\bmod{g(x)} = x+x^2+x^5+x^6-g(x)
= 1 + x^3 + x^5$, or more simply: $s_0 = 110011$.

\section{Bounds on cyclic burst error correcting capability}
Let $C$ have cbecc of $t$. First observation: If $d(C) = d$,
then $t\geq \floor{\frac{d-1}{2}}$
\begin{proof}
    Let $\epsilon = \celling{\frac{d-1}{2}}$. Since $C$ can
    correct any error of with $\leq\epsilon$, all words of
    weight $\leq\epsilon$ are in different cosets of $C$.
    In particular, any error of burst length $\leq\epsilon$ will
    have weight $\leq\epsilon$, so all of these lie in different
    cosets of $C$, so $t\geq\epsilon$.
\end{proof}

Second observation: $t\leq n-k$. So in particular,
$e = **** 0....0$ represents a different coset. Hence
$q^t\leq$ number of cosets of $C$.
We can do better.
\begin{thm}
    \begin{align*}
        t\leq \frac{n-k}{2}
    \end{align*}
\end{thm}

Note: a $(15,9)$-cyclic code has $t\leq \frac{15-9}{2} = 3$.

\subsection{Interleaving}
Idea: we can increase the cyclic burst error correcting
capability $t$ of a code $C$ as follows:

\begin{thm}
    Let $C$ be an $(n,k)$-code over $\mathbb{F}$ with
    c.b.e.c.c capability $t$. Let $C^*$ be the
    $ns, ks$-cyclic code obtained by interleaving $C$ to
    depth $s$. Then $C^*$ has c.b.e.c. capability such that
\end{thm}

Question: If $C$ has generator polynomial $g(x)$, what is gen.
polynomial of $C^*$?

We have $g(x)\mid x^n-1, deg(g) = n-k$, so
$g*(x)\mid x^{ns}-1, deg(g^*)=ns-ks = s(n-k)$.
Answer: $g*(x) = g(x^s)$

\section{BCH Code}
Given $n$ and $\delta$, these codes are designed as cyclic $(n,k)$-codes with
length $n$, and with distance $d\geq\delta$. ($\delta$ is the designed distance).
These are used in CDs, for example, where scratches or other problems might
cause errors.

We need to study minimal polynomials in finite fields to define BCH codes.

\subsection{Minimal Polynomials}
Suppose $\mathbb{F}$ is a finite field of characteristic $p$
\begin{defn}
    Given $\alpha\in\mathbb{F}$, there is a unique polynomial $f(x)\in\mathbb{F}_p[x]$
    of minimum degree which is monic and satisfies $f(\alpha)$. This is called the
    minimal polynomial of $\alpha$ (over $\mathbb{F}_p$).
\end{defn}

\begin{eg}
    $1 + i\in\mathbb{C}$ has what minimal polynomial over $\mathbb{R}$. I want
    $f(z)\in\mathbb{R}[z]$ of minimum degree with $f(1+I) = 0$. By MATH 135,
    $f(z) = (z-(1+i))(z-(1-i)) = (z-1-i)(z-1+1), f(z) = (z-i)^2-i^2 = z^2 - 2z + 2$.
    Observation: we don't only need $\alpha$ as a root, we alsa need its conjugates
    as roots.
\end{eg}

\begin{defn}
    In $\mathbb{F}$ with $char(\mathbb{F}) = p$, the conjugates of $\alpha\in\mathbb{F}$
    are $\alpha, \alpha^p, \alpha^{p^2}...$.
    Note: If $|\mathbb{F}| = q = p^n$, then $\alpha^q = \alpha$ by Fermat little
    Theorem. so the list $\alpha, \alpha^p, \alpha^{p^2}...,\alpha^{p^{n-1}}$ is
    complete, and after it will repeat. The might already be repeatition!
\end{defn}

So will oftern let the following denote the conjugate without repetition,
$conj(\alpha\) = \{\alpha, \alpha^p, \alpha^{p^2},...,\alpha^{p^{t-1}}\}$

\begin{thm}
    The minimal polynomial of $\alpha\in\mathbb{F}$ is $f_{\alpha} =
    \prod_{\beta\in conj(\alpha)} (x-\beta)
    = (x-\alpha)(x-\alpha^p)....(x-\alpha^{p^{t-1}})$
\end{thm}
$f_\alpha(x)\in\mathbb{F}_p[x]$

\section{March 28, 2018}
For $\beta\in\mathbb{F}_{p^n}$, we have
\begin{align*}
    m_B(z) = \prod_{\gamma\in C(B)}(z-\gamma)\\
    = (z-\beta)(z-\beta^q)(z-\beta^{q^2})...(z-\beta^{q^{t-1}})
\end{align*}
and $m_\beta(z)\in\mathbb{F}_q[z]$

\begin{proof}
    Consider my polynomial, $f(z)\in\mathbb{F}_q[z]$ with $f(\beta) = 0$, so if
    $f(z) = a_0 + a_1z + ... + a_{\delta}z^{\delta}$
    then $a_0 + a_1\beta + ... + a_{\delta}\beta^{\delta} = 0$
    Raise to qth power (i.e. apply Frobenius automorphism).
    $(a_0 + a_1\beta + ... + a_{\delta}\beta^{\delta})^q = 0$,
    $(a_0^q + a_1^q\beta^q + ... + a_{\delta q}\beta^{\delta q} = 0$,
    $(a_0^q + a_1\beta^q + ... + a_{\delta}\beta^{\delta q}) = 0$,
    since $a_i^q = a_i$ because $a_i\in\mathbb{F}_q$, so
    $f(\beta^q) =  0$. By this same result, since $\beta^q$ is a root of $f(z)$, so
    is $(\beta^q)^q = \beta^{q^2}$. And again, so are
    $\beta^{q^3}, \beta^{q^4},...$ So all conjugates of $\beta$ are roots.
    $m_B(z) = (z-b)(z-b^q)(z-b^{q^2})...(z-b^{q^{t-1}}) \div f(z)$
    So if $m_b(z)$ shown to be $\mathbb{F}_q[z]$, then it must in fact be min.
    polynomial. But it is! Idea:
    If $m_B(z) = a_0 + a_1z + ... + a_{t-1}z^{t-1}$, then
    $m_B(z)^q = a_0^q + a_1^qz^q + ... + a_{t-1 q}z^{t-1 q}$,
    and also,
    \begin{align*}
        (m_b(z))^q = (z-b)^q(z-b^q)^q...(z-b^{q^{t-2}})(z-b^{q^{t-1}})\\
        (m_b(z))^q = (z^b-b^q)(z^q-b^{q^2})...(z^q-b^{q^{t-1}})(z^q-b^{q^{t}})\\
    \end{align*}
    But in fat, the list of conjusgates "wraps around" back to beginning and
    $\beta^{q^t} = \beta$, and so
    \begin{align*}
        (m_b(z))^q = (z^b-b)(z^q-b^{q})...(z^q-b^{q^{t-1}})\\
        = m_b(z^q) = a_0 + a_1z^q + ... + a_{t-1}z^{(t-1)q}\\
    \end{align*}
    Compairing this, we see that $a_i^q = a_i$ so $a_i\in\mathbb{F}_q$
\end{proof}

Orders of fields elements and factoring $x^n-1$.
\begin{defn}
    Suppose $\beta\in\mathbb{F}_q\setdiff\{0\}$. Then the order of
    $\beta = min\{t\in\mathbb{Z}: t>0, \beta^t = 1\}$
\end{defn}

\begin{thm}
    $ord(\beta) = |\{\beta, \beta^2, ...,\}|$ is the size of the mulplicative group
    generated by $\beta$.
\end{thm}

\begin{thm}
    For any $n\in\mathbb{Z}$, $\beta^n = 1 \leftrightarrow ord(\beta)\div n$
\end{thm}

\begin{eg}
    What is ord(2) for $2\in\mathbb{F}_7$? In $\mathbb{F}_7,
    2^1 = 2, 2^2 = 4, 2^3 = 1, ord(2) = 3$
\end{eg}

Why are we studying this? To define cyclic codes, we work with factorization of
$z^n-1$, and if $ord(\beta)=n$, then
\begin{thm}
    If $ord(\beta)=n$ for $\beta\in\mathbb{F}_q$, then
    in $\mathbb{F}_q[z]$, we have $z^n-1 = (z-1)(z-\beta)...(z-\beta^{n-1})$
\end{thm}


\begin{thm}
    If $ord(\beta) = n$ for $\beta\in\mathbb{F}_q$, then in $\mathbb{F}_q[z]$, we
    have $z^n-1=(z-1)(z-\beta)...(z-\beta^{n-1})$
\end{thm}

\begin{proof}
    $\beta^i$ for $i = 0, 1, ..., n-1$ gives $n$ distinct roots of $z^n-1$, since
    $(\beta^i)^n -1 = (\beta^n)^i -1 = 1^i - 1 = 0$
\end{proof}

Now, we combine the two idea, we factor $z^n-1$ completely, and then we group
term together into minimal polynomials.

\begin{eg}
    Factor $z^5-1$ completely over $\mathbb{F}_2[z]$ starting point, we'll first
    work over a larger field $\mathbb{F}_{2^4} = \mathbb{F}_{2^4} = \mathbb{F}_2[x]
    \setdiff \langle x^4 + x +1$, and note $\beta = x^3$ has order 5.
\end{eg}

Check $\beta^2 = x^6  x^2 + x^3$ by long division (exercise).
$\beta^3 = x + x^3, \beta^4 = 1+x+x^2+x^3, \beta^5 = 1$. So in
$\mathbb{F}_{16}[z]$, we have
$z^5 - 1 = (z-1)(z-\beta)(z-\beta^2)(z-\beta^3)(z-\beta^4)$
How do we get a factorization over $\mathbb{F}_2$?. We group terms together into
minimal polynomials which will be in base field, $\mathbb{F}_2$:
$z-1 = m_{b^0}(z)$ is already in $\mathbb{F}_2[z]$.
$(z-\beta)(z-\beta^2)(z-\beta^4)(z-\beta^8) = m_{\beta^1}(z)$ is in
$\mathbb{F}_2[z]$. since we can check that all conjugates of $\beta$ are
$C(\beta)=\{\beta, \beta^2, \beta^4,\beta^8 = \beta^3\}$ and these are 4
distinct conjugates since $ord(\beta) = 5$. We compute $m_\beta(z) =
1 + z + z^2 + z^3 + z^4$ (exercise) by expanding expression and simplifying
over $\mathbb{F}_{16}[z] = \mathbb{F}_2[x]\setdiff\langle x^4+x+1\rangle)[z]$

So: General process for factoring $z^n - 1 \in\mathbb{F}_q[z]$
Find $\beta\in\mathbb{F}_{q^m}$ such that $ord(\beta) = n$, so
$z^n -1 = \prod^{n-1}_{i=0} (z-\beta^i)$ (we will talk about how to find $m$ soon)

The roots $1, \beta, \beta^2, ..., \beta^{n-1}$, should be partitioned into sets of
conjugates: $\{1, \beta, \beta^2, ..., \beta^{n-1}\} = \cup_{\gamma\in C} C(\gamma).$
disjoin union.
wher $S = \{1, \beta, ...\beta^{n-1}\}$ so that each $C(\gamma)$ is disjoint, and the
union of the $C(\gamma)$ gives $\{1, \beta, ..., \beta^{n-1}\}$

Then we have $z^n -1 = \prod_{\gamma\in S} m_\gamma(z)$, and we compute each
$m_\gamma(z) = (z-\gamma) (z-\gamma)(z-\gamma^q)(z-\gamma^{q^2)})...$
to get all of the irreducible factors of $z^n-1\in\mathbb{F}_q[z]$.

\begin{eg}
    Factor $z^{15}-1\in\mathbb{F}_2[z]$
    \begin{enumerate}
        \item We check that $x\in\mathbb{F}_2[x]\setdiff\langle x^4+x+1\rangle$ has
            $ord(x) = 15$,
            $x^0 = 1, x^1 = x, ..., x^5 = x + x^2, ..., x^{13} = 1 + x^2 + x^{13},
            x^{14} = 1 + x^3, x^{15} = 1$ (Check that $ord(x) = 15$ as exercise)
        \item So $z^{15} -1 = (z-1)(z-x)(z-x^2)...(z-x^{14})$, and we group
            conjugates together
            $m_{x^0}(z) = (z-1), m_{x^1}(z) = (z-x)(z-x^2)(z-x^4)(z-x^8)$, since
            next term repeats $z-x^{16} = z-x$ since $x^{15} = 1$. What's missing
            $(z-x^3)$ for example, $m_{x^3}(z) = (z-x^3)(z-x^6)(z-x^{16})(z-x^9)$
            $m_{x^5}(z) = (z-x^5)(z-x^{10})$
    \end{enumerate}
\end{proof}

\begin{eg}
    If we want to factor $z^{16}-1\in\mathbb{F}[x]$
    \begin{enumerate}
        \item Find $\alpha$ with $ord(\alpha) = 15$ in $\mathbb{F}_{2^m}$
            for some $m$. If we take $m=4$, $\mathbb{F}_{2^4}
            = \mathbb{F}_2[\alpha]\setminus\langle\alpha^4+\alpha+4\rangle,
            z^{15}-1 = (z-\alpha^0)(z-\alpha^1)....(z-\alpha^{14})$.
        \item Group conjugate terms:
            $m_{\alpha_0}(z) = (z-\alpha^0) = z-1$
            Now we account for $(z-\alpha^1)$.
            $m_{\alpha^1}(z) = (z-\alpha^1)(z-\alpha^2)(z-\alpha^4)(z-\alpha^8)$
            Since conjugation mpa is raising to power 2. Using modulo $\alpha^4
            +\alpha + 1$ ... Find $m_{\alpha}(z), m_{\alpha^2}(z), ...$
        \item $z^{15}-1 = (1+z)(1+z+z^4)(1+z+z^2+z^3+z^4)(1+z+z^2)(1+z^3+z^4)$
    \end{enumerate}
\end{eg}

Finishing:
\begin{enumerate}
    \item How do we find $\beta$ of order $n$?
    \item How do we bound the distance of $c$ in terms of the generator polynomial
        $g(z)$?
    \item Construct BCH codes!
\end{enumerate}

To Adress this, we look at the group structure of $\mathbb{F}_q^x$ which is the
group on the set $\mathbb{F}_q^x\setminus\{0\}$ with the multiplication operation
so $|\mathbb{F}_q^x| = q-1$.

\begin{thm}
    $\mathbb{F}_q^x$ has some element $\alpha\in\mathbb{F}^x_q$ such that
    $ord(\alpha) = q-1$ so $\mathbb{F}_q^x = \{\alpha^0,...,\alpha^{q-2}\}$ and
    $\alpha^{q-1} = 1$
\end{thm}

\begin{eg}
    $\mathbb{F}_{2^4}$ had $\alpha$ as a generator. So to find $\beta$ of order $n$
    is simple when $n=q^m-1$ for a power of $q$ minus!! We look in $\mathbb{F}_{q^m}$
    for a generator. What do we do otherwise?

    To find $\beta$ of order $n$, assuming $gcd(n, q) = 1$, we do
    \begin{enumerate}
        \item Find $m$ such that $n\mid q^m-1$
        \item Find a generator $\alpha$ of $\mathbb{F}^x_{q^m}$
        \item $\beta = \alpha^{\frac{q^m-1}{n}}$ has order $n$
    \end{enumerate}
\end{eg}

\begin{eg}
    Find $\beta$ of order 5 to factor $z^5-1\in\mathbb{F}_2[z]$.
    \begin{enumerate}
        \item We want to find $m$ such that $5\mid 2^m -1$, so
            $m=4$ works: $s\mid 2^4-1  = 15$
        \item Find $\alpha\in\mathbb{F}_{2^4}$ of order 15.
            $\alpha\in\mathbb{F}_2[\alpha]\setdiff\langle
            1+\alpha+\alpha^4\rangle$
        \item Take $\beta = \alpha^{\frac{15}{5}} = \alpha^3$
            has order 5. Why? $\beta = \alpha^3 \ne 1$, since
            $ord(\alpha) = 15, \beta = \alpha^6 \ne 1,
            \beta^3 = \alpha^6\ne 1, \beta^4 = \alpha^9\ne 1,
            \beta^4 = \alpha^{12} \ne 14$, but $\beta^5 = \alpha^{15}$
            So $ord(\beta) = 5$. Then to factor $z^5 - 1$, group them
            into conjugate.
    \end{enumerate}
\end{eg}

\begin{defn}
    The cyclotomic coset of $q\bmod{n}$ containing $i$ is
    $C_i = \{i, iq, iq^2, ..., iq^{t-1}\}$ all in modulus $n$
    where $t$ is the smallest positive integer such that
    $iq^t\equiv i\bmod{n}$
\end{defn}

Fact: The irreducible factors of $z^n-1\in\mathbb{F}_q[z]$ correspond
exactly to the cyclotomic cosets and the num. of elments in coset is
the degree.

\begin{eg}
    The cyclotomic cosets of $2\mod 15$ are:
    $C_0 = \{0\}, C_1 = \{1, 2, 4, 8\},
    C_3 = \{3, 6, 12, 9\}, C_5 = \{5, 10\},
    C_7 = \{7, 14, 13, 11\}$. These correspond to the factors.
    $z^{15}-1  = (1+z)(1+z+z^4)(1+z+z^2+z^3+z^4)(1+z+z^2)(1+z^3+z^4)$
    Find minimal polynomial, then gerator factoring....
\end{eg}

BCH Bound
\begin{thm}
    Suppose $g(z)\mid z^n-1$ is the generator polynomial for a cylic code
    over $\mathbb{F}_q$. If $\beta\in\mathbb{F}_{q^m}$ has order $n$, and
    there are $\delta - 1$ consecutive powers of $\beta$ that are roots of
    $g(z)$ then $d(c)\geq\delta$. Explicitly,
    $\beta^a, \beta^{a+1}, \beta^{a+2}, ..., \beta^{a+\delta-2}$ are all
    roots of $g(z)$, then $g(z)\geq\delta$
\end{thm}

\begin{eg}
    How do I find a code of distance $\geq 3$ of length 15 over $\mathbb{F}_{2}$?
\end{eg}

Solution 1: Choose $g(z)\mid z^{15}-1$ such that $\beta^0, \beta^1, \beta^2, \beta^3$
are all roots! By BCH bound, $d(C) \geq 5$.

In fact, the minimal degree $g$ with those roots is
$g(z) = m_{\beta^0}(z)m_{\beta^1}(z)m_{\beta^3}(z) = (1+z)(1+z+z^4)(1+z+z^2+z^3+z^4)$
of degree $9 = n-k$, so $n=15, k=6$, $g(z)$ generates a $(15,6)$-cyclic code of
distance $\geq 5$.


Solution 2: We can do better! $\beta^1, \beta^2, \beta^3, \beta^4$ are roots of
$g(z) = m_{\beta^1}(z)m_{\beta^4}(z)$. Now we have $k=7$, $C$ is a $(15, 7)$-code.

\begin{defn}
    A BCH code of designed distance $\delta$ and length $n$ over $\mathbb{F}_q$ has
    generator polynomial: $g(z) = LCM\{m_{\beta^i}(z): a\leq i\leq a+\delat+2\}$ for
    some integer $a$
\end{defn}

Goals in designing BCH codes: We want $deg g$ to be small, since $k=n-deg g$. Hence
smaller $deg g$ means greater information rate!. So we want a collection of
$\delta-1$ consecutive integers that all lie in a "small" collection of cyclotomic
cosets

\end{document}
